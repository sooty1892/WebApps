\documentclass{article}
\usepackage{graphicx}

\title{WebApps Project - Music Collaboration}
\author{Ashley Hemingway, Charchris Sloan, \\ David Liskevich, Edwin Kamulegeya \\ g1327109}
\date{\today}

\begin{document}
\maketitle

\section{App Description}

\section{User Interactions}
Our proposed web app has multiple user interactions which are both vital to the functionality of the website and the user experience. Obviously, the main user interaction comes with the ability for users to collaborate and share in order to compose music together. Further features include the ability to share music among users and to other websites (including soundcloud, Facebook and twitter).

\section{Implementation Languages}

\subsection{HTML/CSS}
HTML and CSS were an obvious choice. Their popularity is down to the fact that they are the best basis to design a website on. HTML is offers all of the functionality required, and combined with CSS to easily to change the look of the website, it is there perfect choice. HTML also neatly interacts with JavaScript and PHP which is a massive advantage. Furthermore, the newly updated HTML5 offers even more functionally like 'drag and drop' and embedded audio which are both requirements of the web app.

\subsection{JavaScript/jQuery}
JavaScript enables us to add better usability to the website while also keeping things simple. It is widely supported and offers a vast array of frameworks to help us, most noticeably jQuery. It will become very useful when implementing such features like form validation and smoothing easing effects to make the website more user friendly. Furthermore, JavaScript is a requirement when using Ajax, which we will discuss later.

\subsection{PHP}
Although there are many options to code the back-end in like Node, Django and Ruby, we decided to choose PHP. The main reason behind this was that PHP is widely used and it is therefore very stable, with many tutorials and supporting libraries. Since none of us have any experience with  developing the back-end of a web app, we decided that this would give us the best learning experience.

\subsection{Ajax}
Ajax is a necessary component to our web app since it requires content to be pulled from the database once it arrives. This is useful for features like the user chat and the constant updating of the top featured music. There aren't many alternatives to Ajax available, and since our app doesn't require an immediate update (i.e. it can wait 0.5 seconds) it is the best technology at our disposal.

\section{Group Structure}
As expected, we split the project into the front-end and back-end development. The back-end is mainly handled by Ashley Hemingway, who will be designing the database structure and writing the database interaction code in PHP. Ashley will be partly supported by Charchris Sloan, who will be predominately designing the user interface, but will also help with the back-end when required. The front-end will be developed by David Liskevich, Edwin Kamulegeya and helped by Charchris. Their main job will be to design the webpages and implement them.

\end{document}